La personalità è un fattore chiave che influenza le interazioni, i comportamenti e le emozioni delle persone. Al giorno d'oggi, essa viene considera come uno degli argomenti di ricerca più influenti in psicologia.
\\
La crescente immersione negli ambienti digitali e la diffusione dei social network come mezzo di comunicazione, contribuisce alla creazione di un enorme quantità di dati o Big Data, aprendo la necessità allo sviluppo di modelli automatici in grado di leggere con precisione l'essenza degli individui basandosi esclusivamente sulla scrittura.
\\
L'esigenza di produrre analisi sempre più velocemente ci obbliga a sviluppare metodi meccanici per selezionare e interpretare i dati favorendo la ricerca nel campo dell'apprendimento automatico o \emph{machine learning} \cite{samuel1959some}.
\\
Il Data Mining è un nuovo approccio che consiste nell'individuazione d’informazioni significative tramite l'applicazione di algoritmi in grado di individuare le associazioni ``nascoste'' tra di esse  \cite{chakrabarti2006data,franklin2005elements}. 
\\
Una sua forma particolare è il Text Mining, nell'ambito del quale si sono sviluppate metodologie che consentono ai computer di confrontarsi con il linguaggio umano, di elaborarlo e comprenderlo \cite{tan1999text}.
\\
{\color{blue}L'interesse nello studio delle informazioni digitali e le abilità necessarie per farlo non vanno di pari passo tra gli scienziati sociali. Di conseguenza, tale ricerca viene sempre più ceduta a scienziati e ingegneri informatici, offrendo un grande opportunità alle scienze sociali e facilitando la scoperta di modelli che potrebbero non essere evidenti in campioni più piccoli. (?) offre l'opportunità/può essere l'occasione per instaurare una collaborazione interdisciplinare}
\\
La maggior parte degli attuali studi automatici di rilevamento della personalità si sono concentrati sulla teoria dei Big Five come quadro per studiare le caratteristiche intrinseche dell'essere umano \cite{barrick1991big}.
Secondo questo modello esistono cinque dimensioni fondamentali dei tratti sottostanti che sono stabili nel tempo e condivisi a livello interculturale. Le cinque dimensioni, note come i ``Big Five'', sono Openness (apertura all'esperienza), Conscientiousness (coscienziosità), Extraversion (estroversione), Agreeableness (gradevolezza), Neuroticism (nevroticismo), riconosciuti dall'acronimo OCEAN.
\\
I precedenti modelli di previsione della personalità si sono concentrati sull'applicazione di tecniche generali di apprendimento automatico e reti neurali per predire i tratti di personalità Big Five dai post sui social media. 
\\
Sviluppare un modello accurato e aprire questa domanda di ricerca avrebbe implicazioni significative nella business intelligence, nell'analisi della compatibilità delle relazioni e in altri campi della sociologia.

\section*{Struttura della tesi}
{\color{green}Di seguito si passano in rassegna gli argomenti affrontati capitolo per capitolo.

\begin{itemize}
	\item [\setfont{\bfseries Nel  capitolo \nameref{chap:contesto}}] vengono introdotti i concetti teorici alla base del lavoro. Successivamente verranno descritti gli strumenti 
	\item [\setfont{\bfseries Nel capitolo \nameref{chap:RetiNeurali}}] si descriveranno le componenti principali delle reti neurali e in particolare quelle convoluzionali. 
	\item [\setfont{\bfseries Nel capitolo \nameref{chap:formulazione}}] verrà definito il problema e i suoi approcci risolutivi.
	\item [\setfont{\bfseries Nel capitolo \nameref{chap:esperimenti}}] verranno presentati gli esperimenti effettuati e i rispettivi risultati.
	\item [\setfont{\bfseries Nel capitolo \nameref{chap:conclusioni}}] verranno esposte le considerazioni finali.
\end{itemize}
esporre
parlare
presentare

\section*{Motivazioni}
\label{sec:motivazione}
\addcontentsline{toc}{section}{Motivazione}

La natura di questo progetto di tesi è altamente sperimentale. Le motivazioni che hanno portato alla realizzazione di questo lavoro sono la presentazione di analisi dettagliate sull'argomento, in quanto allo stato attuale non esistono importanti indagini di questo tipo.

\section*{Contributi}
\label{sec:contributi}
\addcontentsline{toc}{section}{Contributi}

I dati che verranno utilizzati per definire lo spazio semantico e testare la sua funzionalità sono messi a disposizione da Yelp Dataset Challenge, che contiene \numprint{5200000} recensioni relative a \numprint{174000} attività commerciali di 11 aree metropolitane nel mondo. 
}