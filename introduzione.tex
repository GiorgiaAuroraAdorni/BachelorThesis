La personalità è un fattore chiave che influenza le interazioni, i comportamenti e le emozioni delle persone. Al giorno d'oggi, essa viene considera come uno degli argomenti di ricerca più influenti in psicologia.

La crescente immersione negli ambienti digitali e la diffusione dei social network come mezzo di comunicazione, ha contribuito alla creazione di un enorme quantità di dati o Big Data, aprendo la necessità allo sviluppo di modelli automatici in grado di leggere con precisione l'essenza degli individui basandosi esclusivamente sulla scrittura.

L'esigenza di produrre analisi sempre più velocemente ha imposto lo sviluppo di metodi meccanici per selezionare e interpretare i dati, favorendo la ricerca nel campo dell'apprendimento automatico o \emph{machine learning} \cite{samuel1959some}.

Nello specifico, il \emph{Data Mining} è un approccio che consiste nell'individuazione d’informazioni significative tramite l'applicazione di algoritmi in grado di determinare le associazioni ``nascoste'' tra di esse  \cite{chakrabarti2006data,franklin2005elements}. 

Una sua forma particolare è il \emph{Text Mining}, nell'ambito del quale si sono sviluppate metodologie che consentono ai computer di confrontarsi con il linguaggio umano, di elaborarlo e comprenderlo \cite{tan1999text}.

L'interesse nello studio delle informazioni digitali e le abilità necessarie per farlo non sempre coincidono tra gli scienziati sociali. Di conseguenza, tale ricerca viene generalmente affidata a scienziati e ingegneri informatici, facilitando la scoperta di modelli che non sarebbe possibile individuare ed offrendo l'opportunità di instaurare collaborazioni interdisciplinari.
\\

La maggior parte degli attuali studi automatici di rilevamento della personalità si sono concentrati sulla teoria dei \emph{Big Five} come quadro per studiare le caratteristiche intrinseche dell'essere umano \cite{barrick1991big}.
Secondo questo modello esistono cinque dimensioni fondamentali dei tratti, stabili nel tempo e condivisi a livello interculturale. Le cinque caratteristiche, note appunto come i ``Grandi Cinque'', sono Openness (apertura all'esperienza), Conscientiousness (coscienziosità), Extraversion (estroversione), Agreeableness (gradevolezza), Neuroticism (nevroticismo), riconosciuti dall'acronimo OCEAN.

Sviluppare un modello accurato e aprire questa domanda di ricerca avrebbe implicazioni significative in diversi ambiti della sociologia, ma non solo.

\section*{Struttura della tesi}
Di seguito si passano in rassegna gli argomenti affrontati capitolo per capitolo.

\begin{itemize}
	\item [Nel capitolo \setfont{\nameref{chap:contesto}}] vengono introdotti i concetti teorici alla base del lavoro, in particolare viene introdotta la teoria dei Big Five. 
	\item [Nel capitolo \setfont{ \nameref{chap:RetiNeurali}}] si descrivono le principali tecniche di Deep Learning, in particolare soffermandosi sulle architetture adottate negli esperimenti. 
	\item [Nel capitolo \setfont{\nameref{chap:formulazione}}] viene definito il problema ed illustrati approcci e strumenti risolutivi.
	\item [Nel capitolo \setfont{\nameref{chap:esperimenti}}] viene presentata una panoramica degli esperimenti effettuati e una relativa analisi dei risultati.
	\item [Nel capitolo \setfont{\nameref{chap:conclusioni}}] vengono esposte le considerazioni finali.
\end{itemize}

\section*{Motivazioni}
\label{sec:motivazione}
\addcontentsline{toc}{section}{Motivazione}

La natura di questo progetto di tesi è altamente sperimentale. Le motivazioni che hanno portato alla realizzazione di questo lavoro sono la presentazione di analisi dettagliate sull'argomento, in quanto allo stato attuale non esistono importanti indagini di questo tipo.

\section*{Contributi}
\label{sec:contributi}
\addcontentsline{toc}{section}{Contributi}

I dati che verranno utilizzati per definire lo spazio semantico e testare la sua funzionalità sono messi a disposizione da Yelp Dataset Challenge, che contiene \numprint{5200000} recensioni relative a \numprint{174000} attività commerciali di 11 aree metropolitane nel mondo. 
