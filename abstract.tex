Apprendimento  della personalità basato sul linguaggio naturale. \\


La personalità è considerata come uno degli argomenti di ricerca più influenti in psicologia poiché predittiva di molti esiti consequenziali come la salute mentale e fisica, ed è in grado di spiegare il comportamento umano.
Grazie alla diffusione dei Social Network come mezzo di comunicazione, sta diventando sempre più importante sviluppare modelli che possano leggere automaticamente e con precisione l'essenza di individui basandosi esclusivamente sulla scrittura. 
\\\\
In particolare la convergenza tra scienze sociali e informatiche ha portato i ricercatori a sviluppare approcci automatici per estrarre e studiare le informazioni "nascoste" nei dati testuali presenti in rete.
La natura di questo progetto di tesi è altamente sperimentale, e la motivazione alla base di questo lavoro è presentare delle analisi dettagliate sull'argomento in quanto allo stato attuale non esistono importanti indagini di questo tipo.
\\\\
L'obiettivo è identificare un adeguato spazio semantico che permetta di definire la personalità dell'oggetto a cui un determinato testo si riferisce. Punto di partenza è un dizionario di aggettivi che la letteratura psicologica definisce come marker dei cinque grandi tratti di personalità o Big Five.
\\\\
In questo lavoro siamo partiti dall'implementazione di reti neurali  fully-connected come base per capire come i modelli semplici in deep-learning possano fornire informazioni sulle caratteristiche della personalità nascoste. 
\\\\
Infine utilizziamo una classe di algoritmi distribuzionali inventati nel 2013 da \emph{Tomas Mikolov}, che consistono nell'utilizzo di una rete neurale convoluzionale, che impara in modo non supervisionato i contesti delle parole.
In questo modo costruiamo un embedding in cui sono contenute le informazioni semantiche del testo, ottenendo una sorta di “geometria del significato” in cui i concetti sono tradotti in relazioni lineari.
Con quest'ultimo esperimento ipotizziamo che uno stile di scrittura individuale sia in gran parte accoppiato con i tratti della sua personalità.



Learning personality traits from natural language. \\


Personality is considered as one of the most influential research topics in psychology since it predicts many consequential outcomes such as mental and physical health, and is able to explain human behavior.
Thanks to the spread of Social Networks as a means of communication, it is becoming increasingly important to develop models that can automatically and accurately read the essence of individuals relying solely on writing.
\\\\
In particular, the convergence between social sciences and information technology has led researchers to develop automatic approaches to extract and study "hidden" information in textual data present on the web.
The nature of this thesis project is highly experimental, and the motivation behind this work is to present detailed analyzes on the subject, since there are no such important investigations at present.
\\\\
The objective is to identify an adequate semantic space that allows to define the personality of the object to which a given text refers. Starting point is a dictionary of adjectives that psychological literature defines as a marker of the five great personality traits or Big Five.
\\\\
In this work we started from the implementation of fully-connected neural networks as a basis for understanding how simple deep-learning models can provide information on hidden personality characteristics.
\\\\
Finally we use a class of distribution algorithms invented in 2013 by \ emph {Tomas Mikolov}, which consist in the use of a convolutional neural network, which learns the word contexts in a non-supervised way.
In this way we build an embedding in which the semantic information of the text is contained, obtaining a sort of "geometry of meaning" in which the concepts are translated into linear relations.
With this last experiment we hypothesize that a style of individual writing is largely coupled with the traits of his personality.





